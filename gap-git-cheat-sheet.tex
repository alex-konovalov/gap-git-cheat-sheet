\documentclass[8pt,a4paper]{article}
%\usepackage[margin=1cm,landscape]{geometry}
\usepackage[margin=1cm]{geometry}
\usepackage{flowfram}
\usepackage{xcolor}
\usepackage{url}
%\usepackage[draft]{flowfram}
\ffvadjustfalse
\setlength{\columnsep}{1.4cm}
\Ncolumn{2}

\newcommand{\git}[1]{\newline\indent{\tt{#1}}}

\begin{document}

{\small

\centerline{\Large\bf GAP on GitHub: Quickstart}

\vspace{10pt}
\noindent
{\Large\bf Configuration}\\
You only need to do this once:
\git{git config --global user.name "Your Name"}
\git{git config --global user.email you@yourdomain.com}\\
This data ends up in commits, so do it now before you forget!

\vspace{10pt}
\noindent
{\Large\bf Get the GAP source code}
\git{git clone https://github.com/gap-system/gap.git}

\vspace{10pt}
\noindent
{\Large\bf Branch often}

\noindent
A new branch is like an independent copy of the source code.\\
To switch to a new branch before editing, call:
\git{git checkout master} \hfill \textcolor{red}{switch to the starting point}
\git{git branch new\_branch\_name} \hfill \textcolor{red}{create new branch}
\git{git checkout new\_branch\_name} \hfill \textcolor{red}{switch to new branch}\\
Without an argument, the list of branches is displayed:
\begin{verbatim}
git branch
master
* new_branch_name 
\end{verbatim}
where the star marks the current branch.

\noindent
When you finished, \textcolor{red}{delete unused branches}:
\git{git branch -d branch\_to\_delete}


\vspace{10pt}
\noindent
{\Large\bf Who did what?}

\noindent
Each change recorded by git is called a \textcolor{red}{commit}.\\
To examine commit history, call
\git{git show} \hfill \textcolor{red}{show the most recent commit (ie current head)}
\git{git log} \hfill \textcolor{red}{commit list in reverse chronological order}

\vspace{10pt}
\noindent
{\Large\bf What have I done?}

\noindent
Probably the most important command. Example output:
\begin{verbatim}
git status
  On branch new_branch_name        // current branch name
Changes not staged for commit:
  (use "git add <file>..." to update what will be committed)
  (use "git checkout -- <file>..." to discard changes in
   working directory)

        modified: modified_file.g  // file you just edited
        
Untracked files:
  (use "git add <file>..." to include in what will be
   committed)

        new_file.g                 // file you just added
        
no changes added to commit
(use "git add" and/or "git commit -a")
\end{verbatim}

\vspace{10pt}
\noindent
{\Large\bf Preparing your Commit}

\noindent
When ready, tell git which changes do you want to commit:
\git{git add filename} \hfill \textcolor{red}{add particular file}
\git{git add .} \hfill \textcolor{red}{add all modified and new}

\noindent
The status command then lists the staged changes:
\begin{verbatim}
git status
On branch new_branch_name
Changes to be committed:
  (use "git reset HEAD <file>..." to unstage)

       modified: modified_file.g
       new file: new_file.g
\end{verbatim}

\noindent
\vspace{10pt}
{\Large\bf Commit!}

\noindent
The commit command will permanently record the staged changes. 
The new commit becomes the new branch head. 
To {\textcolor{red}{write commit message} in an editor, call
\git{git commit}\\
otherwise just call
\git{git commit -m "My Commit Message"}

\noindent
Commits cannot be changed, but they can be discarded and re-done
with the {\tt --amend} switch. {\bf Never amend commits that you have
already shared with somebody.}

\vspace{9pt}
\centerline{\Large\bf Summary}

\noindent
{\bf Workspace} is the file system: files that you can edit
\git{git add filename} \hfill \textcolor{red}{copy file to staging}
\git{git reset HEAD filename} \hfill \textcolor{red}{copy staged file back}\\
{\bf Staging} is a special area inside the git repository
\git{git commit} \hfill \textcolor{red}{commit all staged files}\\
{\bf Commits} are the permanently recorded history
\git{git checkout -- filename} 
\hfill \textcolor{red}{copy {\tt filename} from \\ \indent \hfill repository to workspace}

\vspace{10pt}
\centerline{\Large\bf Keeping in sync}

\noindent
If you have write access to the GAP repository,
next time before you start any changes you should 
incorporate changes from the master branch of the 
remote repository with
\git{git pull origin master} \hfill \textcolor{red}{fetch and merge}\\
or (to linearise history if you have not pushed changes) with
\git{git pull --rebase origin master} \hfill \textcolor{red}{fetch and rebase}\\
Resolve conflicts, if there are any (see below),
then switch to the top of the branch:
\git{git checkout master} \hfill \textcolor{red}{switch to the starting point}\\
If you don't have write access, see \url{https://help.github.com/articles/syncing-a-fork/} on syncing a fork.

\vspace{5pt}
\centerline{\Large\bf Merging}

\noindent
A commit with more than one parent is a merge commit:
\git{git merge other\_branch}\\
incorporates other branch/commit. If there is no conflict, 
this automatically creates a new merge commit. Otherwise, 
the conflicting regions are marked like this:
\begin{verbatim}
Lines that are either unchanged from a common ancestor, 
or cleanly resolved because only one side changed.
<<<<<<< yours:source_file.g
It's frustrating to resolve conflicts. 
Let's have coffee.
=======
It's easy to resolve conflicts with Git.
>>>>>>> theirs:source_file.g
Another line that is cleanly resolved or unmodified.
\end{verbatim}
Edit as needed; To finish, run one of:
\git{git commit} \hfill \textcolor{red}{commit your merge conflict resolution}
\git{git merge --abort} \hfill \textcolor{red}{ discard merge attempt}

\vspace{10pt}
\centerline{\Large\bf Branch Heads}

\noindent
A git branch is just a pointer to a commit. This commit is called the
branch {\tt HEAD}. You can point it elsewhere with ({\tt --hard}) 
or without ({\tt --soft}, less common) resetting the actual files. That is, the
following discards content of the current branch and makes it
indistinguishable from a new branch that started at {\tt new\_head\_commit}:
\git{git reset --hard new\_head\_commit}

\noindent
There are various ways to specify a commit to reset to:\\
{\tt 994bd8d4b279c549cb01ef4e684c246e555f6bae} \hfill \textcolor{gray}{40-digit sha1}\\
{\tt 994bd8d} \hfill \textcolor{gray}{the first few digits of the sha1}\\
{\tt branch\_name} \hfill \textcolor{gray}{the name of another branch pointing to it}\\
{\tt 4.8.beta0} \hfill \textcolor{gray}{a tag in the GAP git repository}\\
{\tt origin/master} \hfill \textcolor{gray}{the master branch in the remote origin}\\
%{\verb|HEAD~|} \hfill \textcolor{gray}{first parent of the current head}\\
%{\verb|HEAD~2|} \hfill \textcolor{gray}{first parent of the first parent of the current head}\\
%{\verb|HEAD^2|} \hfill \textcolor{gray}{second parent (if exists) of the current head}\\
%{\tt FETCH\_HEAD} \hfill \textcolor{gray}{commit downloaded with the git fetch command}\\

\vspace{8pt}
\centerline{\Large\bf Most important git commands!}

\noindent
Git comes with extensive documentation:
\git{git help -i} \hfill most commonly used Git commands
\git{git help -a} \hfill list of all Git commands
\git{git help -g} \hfill list of available Git guides

\vspace{2pt}
\noindent{\it \tiny
Adapted from \url{http://github.com/sagemath/git-trac-command/raw/master/doc/git-cheat-sheet.pdf}}

}
\end{document}
